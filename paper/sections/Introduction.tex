\documentclass[../main.tex]{subfiles}

\begin{document}

\section{Introduction}
A sorting network is a formal representation of a sorting algorithm that for any inputs generates
monotically increasing outputs. A sorting network is formed by n channels each of them carrying one input,
which are connected pairwise by comparators. A comparator compares the inputs from it's 2 channels and 
outputs them sorted to the same 2 channels. A comparator network is a sorting network if for any 
input sequence the output is always the sorted sequence. What makes sorting networks special is their
high parallelization capacity, we can create parallel layers of comparators as long as none of them is part of
the same input channel at once.

The creation of optimal sorting networks can be divided in 2 tasks. Finding sorting networks with less comparators,
also called size optimization. And finding sorting networks with less parallel layers, also called depth optimization.
In this thesis we will focus in the size optimization.

The search of optimal size sorting networks involves to test all comparator networks of a giving size. 
For example to prove the optimality of the sorting network
with 11 inputs and 35 comparators we should consider 55 = (11 x 10)/2 possibilities to place each comparator in 2 out of 11 channels. Therefore the search space is of $55^{35} \approx 9 * 10^{60}$ comparator networks. 

This problem can only be addressed by using symmetry breaking rules to trim the search space. For this matter first I implemented the method used in \cite{sortingnineinputs} called generate and prune. This method is formed by 2 phases.
In the generate phase, starting with a one comparator network it iteratively creates new networks with one comparator more in all possible positions. In the prune phase the redundant networks (networks equivalent to others in the set) are removed. This way the search space is reduced to $2.2 * 10^{37}$ to around $3.3 * 10^{21}$ for the 9 inputs network.

In the first part of this thesis, focused in improving the implementation using modern programming languages and reducing the memory and CPU consumption. This itself allowed to reproduce the same results than in \cite{sortingnineinputs} with only 10 hours of compute time in a 64 cores computer more modest than the one used in \cite{sortingnineinputs} which took one week of computing. However, when trying to address the next open problem, this itself is not enough. Due to the combinatory explosion it would be necessary around a year of computing time in this computer to find a solution for the 11 input problem. This led to the second part of this thesis where I apply heuristics to the generate and prune method. During the test of heuristic I discovered without prove a method that allows to discover nets with the same size than all the already best size networks until 13 inputs. For 14 inputs and above again the sets are too big to be tested in the available hardware.

The combination of heuristic functions with generate and prune has dealt promising results, finding networks of the same size than the state of the art smallest networks in the interval 3-16. In the following chapters I will state with further details the work performed in this master thesis.

\end{document}