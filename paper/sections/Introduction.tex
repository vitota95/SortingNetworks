\documentclass[../main.tex]{subfiles}

\begin{document}

\section{Introduction}
A sorting network is a data-oblivious sorting algorithm, which means that the order of comparisons to be performed is only determined by the number of inputs to be sorted and not the order of these values. A sorting network is formed by $n$ channels each of them carrying one input, which are connected pairwise by comparators. A comparator compares the inputs from its two channels and 
outputs them sorted to the same channels. Such a sequence of comparators is called a comparator network. A comparator network is a sorting network if for any input, the output is always the sorted sequence. What makes sorting networks special is their high parallelization capacity, we can create parallel layers of comparators as long as none of them are part of
the same input channel at once.

Optimizing sorting networks can be done in two ways. Finding sorting networks with fewer comparators,
also called size optimization. And finding sorting networks with fewer parallel layers, also called depth optimization.
In this thesis we will focus on size optimization.

The only known way of finding optimal size sorting networks involves testing all comparator networks of a given size. 
For example to prove the optimality of the sorting network
with 11 inputs and 35 comparators we should consider $55 = (11 \times 10)/2$ possibilities to place each comparator in 2 out of 11 channels. Therefore, the search space is of $55^{35} \approx 9 \times 10^{60}$ comparator networks. 

One of the ways to address this problem is by using symmetry breaking rules to trim the search space. For this purpose, the authors of \cite{sortingnineinputs} proposed a method that they called generate-and-prune. This method is formed by two phases that alternate, starting from an empty comparator network.
In the generate phase, the method iteratively creates new networks with one comparator more in all possible positions. In the prune phase, starting with the output set of the Generate algorithm, the redundant networks (networks equivalent to others in the set) are removed. This way the search space is reduced to $2.2 \times 10^{37}$ to around $3.3 \times 10^{21}$ for the 9 inputs network.

In the first part of this thesis, I focused on implementing and improving generate-and-prune using modern programming languages and reducing the memory and CPU consumption. This itself allowed to reproduce the same results as in \cite{sortingnineinputs} (demonstrate that the minimum number of comparators to sort 9 inputs is 25), with only 10 hours of compute time in a 64-core computer, while in \cite{sortingnineinputs} they used 144 threads and took over one week of computing. However, when trying to address the next open problem, minimum size for 11-input sorting networks, this itself is not enough. Due to the combinatory explosion it would be necessary around a year of computing time in this computer to find a solution for the 11-input problem. This led to the second part of this thesis where I apply heuristics to the generate-and-prune method in order to find sorting networks better than the best known. One of the heuristics tested is able to discover networks with the same size than all the already best known sizes for networks until 13 inputs. For 14 inputs and above again the intermediate sets generated during generate-and-prune are too big to manage using the available hardware.

The combination of heuristic functions with generate-and-prune has dealt promising results, finding networks of the same size than the state of the art smallest networks in the interval 3-16. 

During the next sections I will present the work performed in this master thesis. In section 2 I will give some preliminaries in sorting networks and introduce the notation, in section 3 I will explain how to implement the generate-and-prune algorithm from \cite{sortingnineinputs}. Sections 4 and 5 explain the techniques and heuristics to lower the search for the optimal-size problem and are the main focus of this master thesis.

The source code used for the implementation of this thesis can be accessed in: https://github.com/vitota95/SortingNetworks.
\newpage
\end{document}