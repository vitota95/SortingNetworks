\documentclass[../main.tex]{subfiles}
\graphicspath{{\subfix{images/}}}

\begin{document}
	\section{Heuristics}
	In this section we study heuristic methods that can reduce the population size after generate-and-prune while still leading to promising results. In particular three methods were used. The first two methods discard networks based in some characteristics in their outputs sets. And the last one avoids the permutations check in the prune phase, setting as subsumed any network that arrives to that stage. Combinations of these heuristics have dealt promising results finding minimal sorting networks.
	
	\subsection{Remove networks with more outputs heuristic}
	A sorting network of size $n$ has $n+1$ binary outputs. Consequently, we can say that networks containing less outputs order "better" than those containing more. In this heuristic we order the set of networks by their number of outputs and remove the ones with larger number of outputs until a given population size. This heuristic is one of the most common in the literature such as \cite{Sekanina2005} and \cite{SortingNetworkIsomorphism}.
	
	\subsection{Bad positions heuristic}
	This heuristic is defined in \cite{FRASINARU2019447}. It takes a comparator network and gives it a score based on the number of 'bad positions' that its outputs contain. A bad zero is defined as follows:
	
	Let $C$ be a comparator network of with $n$ inputs, if $C$ is a sorting network then for any input sequence $x$ that contains $p$ zeroes the output will be the sequence that contains zeroes in the first $p$ positions and ones in the rest $n-p$ positions. We call bad position to those that contain a zero in a place where a 1 should be. The evaluation function is as follows:
	
	$f(C)=\frac{1}{(n+1)\times(2^n-1)}\times(2^n \times |bad(C)| + |outputs(C)| - n -1)$
	
	\subsection{Prune with no permutations heuristic} \label{noPermutationsHeuristic}
	The last heuristic that we tried is pruning every network which subsumption is not discarded based in the Lemmas of section 5. Given $C_1$ and $C_2$ comparator networks, when checking if $C_1$ subsumes $C_2$ if the subsumption is not discarded applying the Lemmas of section 5, instead of trying to find a permutation $\pi$ that $\pi(outputs(C_a)) \subset outputs(C_b)$ we directly mark $C_b$ as subsumed. This way we reduce the subsume operation complexity from $O(n! \times f(n))$ to $O(1)$. When applying this rule alone, the sizes of the biggest sets are reduced up to a around 0.05\% of those of the subsumption test studied before. This heuristic alone was able to find always a sorting network with the already best known sizes in the interval $2...13$ and in combination with the Remove big outputs heuristic we found networks with the best known sizes with inputs up to $n=16$. In Table \ref{table:noPermutations} we compare the sets after prune for $n=9$ of both implementations. To put it into perspective, while the traditional approach took 10 hours and a half in a computer with 126 processors to deal a result for $R{_9^{25}}$, with this heuristic we get a sorting network of size 25 in one second using a laptop.  
	
	\begin{table}[H]
		\hspace*{-1cm}
		\begin{tabular}{|c c c c c c c c c c c c c c|}
			\hline
			$k$ & 1 & 2 & 3 & 4 & 5 & 6 & 7 & 8 & 9 & 10 & 11 & 12 & 13  \\
			\hline\hline
			$R{^9_k}$& 1 & 3 & 7 & 20 & 59 & 208 & 807 & 3415 & 14,343 & 55,951 & 188,730 & 480,322 & 854,638 \\ 
			\hline
			$S{^9_k}$ & 1 & 2 & 3 & 7 & 13 & 22 & 41 & 77 & 136 & 229 & 302 & 403 & 531 \\
			\hline
		\end{tabular}
		\caption{Compared sizes of filter sets for $n=9$}
		\label{table:noPermutations}
	\end{table}
	\begin{table}[H]
		\hspace*{-1cm}
		\begin{tabular}{|c c c c c c c c c c c c c|}
			\hline
			k & 14 & 15 & 16 & 17 & 18 & 19 & 20 & 21 & 22 & 23 & 24 & 25  \\
			\hline\hline
			$R{^9_k}$ & 914,444 & 607,164 & 274,212 & 94,085 & 25,786 & 5699 & 1107 & 250 & 73 & 27 & 8 & 1 \\ 
			\hline
			$S{^9_k}$ & 586 & 570 & 519 & 413 & 314 & 230 & 179 & 123 & 57 & 24 & 8 & 1 \\
			\hline
		\end{tabular}
	\end{table}
	
	We actually suspect that this heuristic could be complete, the networks surviving this prune create a minimum filter for the optimal size problem, which bring us to the next conjecture:
	
	\begin{conjecture}
		Given two comparator networks $C_a$ and $C_b$ with $n$ channels in which we are checking if $C_a$ subsumes $C_b$. If Lemmas \ref{lemmaSubsume1} and \ref{lemmaSubsume2} do not discard the subsumption then $C_a$ subsumes $C_b$.
	\end{conjecture}

	\subsection{Heuristics results}
	In this section we state the results of using the heuristics and combinations of them. Both remove networks with more outputs and bad positions heuristic have similar results and in combination with prune with no permutations lots of networks were found with relatively small population sizes. In the next tables we will find comparisons of the proposed heuristic with respect to the best known sorting network sizes, using different population sizes. The information for the sizes of the networks has been extracted from \cite{bertdobbelaere}.
	
	In the next table we can see the results obtained when combining the heuristics of remove networks with more outputs and prune with no permutations. 
	\begin{table}[H]
		\begin{center}
		\begin{tabular}{|c | c c c c c c c c|} 
			\hline
			inputs & 9 & 10 & 11 & 12 & 13 & 14 & 15 & 16  \\ [0.5ex] 
			\hline\hline
			$S(n)$ population 100 & \textbf{25} & 30 & \textbf{35} & \textbf{39} & 48 & 53 & 59 & 63 \\ [1ex]
			\hline
			$S(n)$ population 250  & 25 & 30 & 35 & 39 & 48 & 53 & 59 & 63 \\  [1ex] 
			\hline
			$S(n)$ population 500 & 25 & \textbf{29} & 35 & 39 & 48 & 53 & 59 & 61 \\  [1ex] 
			\hline
			$S(n)$ population 1000  & 25 & 29 & 35 & 39 & 48 & \textbf{51} & 59 & 61 \\  [1ex] 
			\hline
			$S(n)$ population 5000  & 25 & 29 & 35 & \textbf{39} & 47 & 51 & 57 & \textbf{60} \\  [1ex] 
			\hline
			Best known size & 25 & 29 & 35 & 39 & 45 & 51 & 56 & 60 \\  [1ex] 
			\hline
		\end{tabular}
		\end{center}	
		\caption{Best sizes combining networks with more outputs and prune with no permutations}
	\end{table}

	The bad positions heuristic combined with prune without permutations have accomplished poorer results in comparison. With populations up to 5000, we could only get the best results for the networks from 9 to 12 inputs, while the best sizes for the bigger input networks are far from the state-of-the-art best ones.
	\begin{table}[H]
		\begin{center}
				\begin{tabular}{|c | c c c c c c c c|} 
				\hline
				inputs & 9 & 10 & 11 & 12 & 13 & 14 & 15 & 16  \\ [0.5ex] 
				\hline\hline
				$S(n)$ population 100 & \textbf{25} & 31 & 37 & 46 & 52 & 62 & 72 & 82 \\ [1ex]
				\hline
				$S(n)$ population 250  & 25 & \textbf{30} & 37 & 42 & 51 & 60 & 70 & 80 \\  [1ex] 
				\hline
				$S(n)$ population 500 & 25 & 30 & 36 & 41 & 51 & 59 & 69 & 78 \\  [1ex] 
				\hline
				$S(n)$ population 1000 & 25 & 30 & \textbf{35} & 41 & 49 & 58 & 67 & 77 \\  [1ex] 
				\hline
				$S(n)$ population 5000 & 25 & 29 & 35 & 39 & 48 & 54 & 64 & 74 \\  [1ex] 
				\hline
				Best known $S(n)$ & 25 & 29 & 35 & 39 & 45 & 51 & 56 & 60 \\  [1ex] 
				\hline
			\end{tabular}
		\end{center}
		\caption{Best sizes combining bad positions and prune with no permutations}
	\end{table}
	
	When it comes to bigger instances of the problem, we tried input sizes of 17, 18, 19 and 21 mixing prune with no permutations and remove networks with more outputs heuristics.
	
	We were able to find a network with 19 inputs and 86 comparators, same size than \cite{valsalam:jmlr13} using a maximum population size of 50,000 networks and that would have been record in 2013. Since then another network with 85 comparators has been found.
	
	In the following table we can see the best network sizes found during this master thesis:
	
	\begin{table}[H]
		\begin{center}
			\begin{tabular}{|c | c c c c c|} 
				\hline
				inputs & 17 & 18 & 19 & 20 & 21  \\ [0.5ex] 
				\hline
				Best found size & 72 & 78 & 86 & - & 104 \\ [1ex]
				\hline
				Best known size & 71 & 77 & 85 & 91 & 100 \\ [1ex]
				\hline
			\end{tabular}
		\end{center}	
		\caption{Best sizes obtained in big networks combining heuristics}
		\label{table:sizesHeuristics}
	\end{table}

	When it comes to use the prune with no permutations heuristic the algorithm used the following times in finding a sorting network for values of $n$ in the interval $[7..13]$. The next table shows the times spent to find a sorting network for the prune with no permutations heuristic.
	
	\begin{table}[H]
		\begin{center}
			\begin{tabular}{|c | c c c c c c c|} 
				\hline
				inputs & 7 & 8 & 9 & 10 & 11 & 12 & 13  \\ [0.5ex] 
				\hline
				Time in seconds & 0.1 & 0.2 & 1 & 5 & 63 & 251 & 21823 \\ [1ex]
				\hline
			\end{tabular}
		\end{center}	
		\caption{Best sizes obtained in big networks combining heuristics}
		\label{table:timeHeuristics}
	\end{table}
	
	 As it is noticeable, the computation time scalates quickly in bigger instances. While for $n=12$ the process finishes in the order of minutes, for $n=13$ it already takes around 6 hours of computation. However, we believe that using this solution with the appropriate hardware and in combination with other heuristics, could find better networks for the bigger instances.

	\newpage
\end{document}